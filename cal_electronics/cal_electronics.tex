%% cal_electronics.tex  Calorimeter  Electronics Eric, Chi 20p    
The sPHENIX reference design for electronics 

\section{Optical Sensors}


\subsection{Device Characteristics}
For both the electromagnetic and hadronic calorimeters, we are
currently considering as sensors 3\,mm$\times$3\,mm silicon photomultipliers
(SiPMs), such as the Hamamatsu S10362-33-25C MultiPixel Photon
Counters (MPPC).  These devices have 40,000 pixels, each
15\,$\mu$m$\times$ 15\,$\mu$m. Any SiPM device will have an intrinsic
limitation on its dynamic range due to the finite number of pixels,
and with over 40K pixels, this device has a useful dynamic range of
over 10$^4$.  The saturation at the upper end of the range is
correctable up to the point where all pixels have fired.  The photon
detection efficiency is $\sim25\%$ and it should therefore be possible
to adjust the light level to the SiPM using a mixer to place the full
energy range for each tower ($\sim$ 5\,MeV--50\,GeV) in its useful
operating range. For example, if the light levels were adjusted to
give 10,000 photoelectrons for 50\,GeV, this would require only 200 photoelectrons/GeV,
which should be easily achieved given the light level from
the fibers entering the mixer.

\subsection{Neutron Radiation Effects}

\section{Front End Analog Electonics}
\subsection{Pre-Amplifiers}
\subsection{Controller Boards}
\section{Digitizers Electronics}
\subsection{Digitizer Boards}
\subsection{XMIT Boards}
\subsection{Controllers}

 
Once you cite references, delete this text being used
to avoid a BibTeX error~\cite{Adare:2010ux}.